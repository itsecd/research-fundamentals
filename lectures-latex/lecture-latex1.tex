\documentclass[12pt]{article}

\usepackage[english, russian]{babel}
\usepackage{amsmath, amsfonts, amsthm}
\usepackage{graphicx}
\usepackage{float}
\usepackage{multicol}
\addto\captionsrussian{\def\refname{Список использованных источников}}

\begin{document}% \begin - окружение, которое надо закрывать с помощью \end

\begin{titlepage}
\title{Примеры}
\author{Кипкаева Ольга Сергеевна}
\date{сегодня}
\maketitle % Заголовок
\thispagestyle{empty}
\end{titlepage}

\tableofcontents
\newpage

\section{Списки}
\subsection*{С заголовками} % раздел не пронумеровался
\begin{description}
\item[itemize:] пункты помечаются маркерами;
\item[enumerate:] пункты нумеруются;
\item[description:] пункты снабжаются заголовками.
\end{description}

\subsection{Нумерованный}
\begin{enumerate}
\item первый пункт
\item второй пункт
\end{enumerate}

\subsection{Маркированный}
\begin{itemize}
\item первый пункт
\item второй пункт
\end{itemize}

\subsection{Свой}
\begin{itemize}
\item[$+$] первый пункт
\item[$-$] второй пункт
\end{itemize}


\subsection{Вложенный}
\begin{enumerate}
\item Нумеруются
\begin{enumerate}
\item второй уровень вложенности
\end{enumerate}
\item еще один пункт
\end{enumerate}

\subsection{Ненумерованный}
\begin{trivlist}
\item первое
\item второй
\item третий
\end{trivlist}
\newpage

\section{Таблички}
\begin{tabular}{|c|c||}
1 & 5 \\
\hline
2 & 3
\end{tabular}
\\
\\
\\
\begin{tabular}{|c|l|r|p{6cm}|}% c, r, l - выравнивание по центру, по правому краю, по левому краю
\hline
13 & 54 & 60 & 9 \\
\hline
3 & 8 & 4 & 2 \\
\hline
9 & 5 & 5 & 4\\
\hline
\end{tabular}
\newpage

\section{Колонки}
\begin{minipage}[t]{14mm}
Левая колонка узкая
\end{minipage}
\hfill
\begin{minipage}[t]{38mm}
Правая колонка немного шире
\end{minipage}

\columnseprule=0.3pt\columnsep=24pt
\begin{multicols}{3}
    какой-то очень длинный текст про мамонтов и динозавров, которые жили за много лет до нас
\end{multicols}
\newpage

\section{Формулы} % \section - команда, которую не надо закрывать

Пример 1 $\alpha$\\ % не выносится из текста
Пример 2 $$\alpha_1$$ % выносится на середину строки

Пример 3 \[\alpha_2\] % выносится на середину строки

\textbf{Специальные знаки} $\%, \{, \setminus$\\

$$\mathbb R, \mathbf L, \mathcal L, \mathrm G, G$$

$$\epsilon, \varepsilon, \phi, \varphi$$

\subsection{Индексы}
Если индексов несколько, их нужно объединить в группы с помощью скобок $\{\}$\\
Индексы друг над другом
$$a_{(n-1)}^{546}, a^{546}_{45}$$
и нет
$$a^2{}_1{}^4$$

$$a^\prime=a'$$

$$\vec{\dot a}$$

$$\underbrace{
\overbrace{0 1 2 \dots 9}^{10}
\rm A B \dots F
}_{16}$$

\subsection{Корни и дроби}

$$\sqrt[3]{x-y}$$
$$\sqrt{dy} \sqrt{d} \sqrt{y}$$

$$\frac{13}{54}$$

Команда frac может уменьшать дроби, если они находятся в тексте или числителе/знаменателе другой дроби\\
Дробь frac $\frac{13}{54}$ и dfrac $\dfrac{13}{54}$\\
Дробь frac
$$\frac{1}{4+\frac1{1+\frac12}}$$
Дробь cfrac
$$\cfrac{1}{4+\cfrac1{1+\cfrac12}}$$

\subsection{Скобки}

$$\Biggl(
\biggl[
\Bigl\{
\bigl\|
\langle x \rangle
\bigr\|
\Bigr\}
\biggr]
\Biggr)$$

$$\left(\cfrac{1}{4+\cfrac1{1+\cfrac12}}\right)$$ 

Можно поставить только одну скобку
$$\biggl($$

$$\left. \right)$$

\subsection{Промежутки и пробелы}
пмрп \,пм\\
орир\:прмп\\
лои\;иро\\
тот\qquad ма \\

ас\!мапм

\subsection{Операторы}

$$sin^2x+cos^2x=1$$
$$\sin^2x+\cos^2x=1$$

$\lim_{x \to 0}$

$\lim\limits_{x \to 0}$\\

$\int_4^5$

$\int\limits_4^5$

\subsection{Матрицы}

$\begin{matrix}
1 & 0\\ 
2 & 3
\end{matrix}$
$\begin{pmatrix}
1 & 0\\ 
2 & 3
\end{pmatrix}$
$\begin{bmatrix}
1 & 0\\ 
2 & 3
\end{bmatrix}$
$\begin{vmatrix}
1 & 0\\ 
2 & 3
\end{vmatrix}$
$\begin{pmatrix}
1 & \dots & 0\\ 
\vdots & \ddots & \vdots\\
2 & \dots & 3
\end{pmatrix}$


\subsection{Системы и нумерация}
\subsubsection{Ненумерованные}
$$\begin{cases}
gfftfy, & \text{если $x=0$}\\
nmb h
\end{cases}$$

$$\begin{aligned}
gfftfy, & \text{если $x=0$}\\
nmb h
\end{aligned}$$

\subsubsection{Нумерованные}
\begin{align} % выравнивает уравнения по знаку &
67867 &= 65658 & 546 & =5\\
5 &= 45 & 5 & =897
\end{align}

\begin{gather}% выравнивает по центру
gfftfy, \text{если $x=0$} \label{1}\\
nmb hnbn \notag % не нумеруется одно уравнение
\end{gather}

Ссылка на формулу (\ref{1}).

\begin{multline}% разбивает длинную формулу на несколько строк
dgsgdfgf\\
ffgfgf\\
tfttgyg\\
fgdfgfd
\end{multline}

Чтобы пронумеровать не каждое уравнение в системе, а всю систему, нужно в окружении equation записать окружение системы, например cases\\
\begin{equation}
|\sin x|=
\begin{cases}
\sin x, & 0< x <\pi, \\
-\sin x, & \pi< x <2\pi.
\end{cases}
\end{equation}
\newpage

\section{Рисунки}
Ссылка на литературу \cite{g}.

Уточка \\
\includegraphics[width=110mm]{utka.jpg}

\begin{figure}[h] % С параметром [h] латех сам решает, куда поставить картинку, но поставит примерно там, где вы ее написали в коде. Если не хватит места на данной странице, он перекинет картинку на другую страницу, причем текст, написанный после картинки, может остаться на данной странице. То есть, он выведется перед картинкой
\centering{
\includegraphics[width=130mm]{utka.jpg}
}
\caption{Утка плавает где-то}
\label{ут}
\end{figure}


Ссылка на рисунок \ref{ут} ставится той же командой, что и ссылка на формулу.

\begin{figure}[H] % пакет float и параметр [H] дает возможность поставить картинку в той части кода, где вы ее поставили
\centering{
\includegraphics[width=130mm]{utka.jpg}
}
\caption{Утка находится там, куда вы ее поставите}
\label{ут}
\end{figure}
\newpage

\addcontentsline{toc}{section}{Список исп ист}
\begin{thebibliography}{0}
\bibitem{g}
сапспмо
\end{thebibliography}

\end{document}
